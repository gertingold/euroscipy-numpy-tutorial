\documentclass{beamer}
\usepackage[utf8]{inputenc}
\usepackage{arev}
\usepackage{beramono}
\usepackage{fontawesome}
\usepackage[T1]{fontenc}
\usepackage{pifont}
\usepackage{rotating}
\usepackage{listings}
\usepackage{hyperref}

\usetheme{default}
\setbeamertemplate{navigation symbols}
{%
%  \hspace{3em}
%  \vbox{%
%  \hbox{\insertslidenavigationsymbol}
%  \hbox{\insertframenavigationsymbol}
%  \hbox{\insertbackfindforwardnavigationsymbol}
%  \vspace{2em}}
}

\setbeamercolor{alerted text}{fg=red!70!black}
\setbeamercolor{structure}{fg=red!50!black}


\hypersetup{%
  pdftitle={Introduction to NumPy arrays}
  ,pdfauthor={Gert-Ludwig Ingold <gert.ingold@physik.uni-augsburg.de>}
  ,pdfsubject={Tutorial at EuroSciPy 2016, Erlangen 23.8.2016}
  ,pdfkeywords={Python, NumPy, tutorial, EuroSciPy}
}

\lstset{%
  language=Python
  ,basicstyle={\ttfamily},
}

\newfontfamily\DejaSans{DejaVu Sans}
\definecolor{positive}{rgb}{0, 0.5, 0}
\definecolor{negative}{rgb}{0.7, 0, 0}
\newcommand\smiley{\textcolor{positive}{\DejaSans ☺}}
\newcommand\frownie{\textcolor{negative}{\DejaSans ☹}}

\graphicspath{{./images/}}

\begin{document}

\begin{frame}
 \begin{center}
  \structure{\Large\textbf{Introduction to NumPy arrays}}\\[0.3truecm]
  \structure{\large Gert-Ludwig Ingold}

  \vspace{2truecm}
  \faicon{github}\ \ttfamily{\footnotesize https://github.com/gertingold/euroscipy16-numpy-tutorial.git}
 \end{center}
\end{frame}

\begin{frame}[t]{Python comes with batteries included}
 \ding{220} extensive Python standard library

 \vspace{0.3truecm}
 \structure{What about batteries for scientists (and others as well)?}

 \vspace{0.2truecm}
 \ding{220} scientific Python ecosystem

 \begin{center}
  \only<1>{\includegraphics[width=0.7\textwidth]{SciPyEco}}%
  \only<2>{\includegraphics[width=0.7\textwidth]{SciPyEco_NumPy}}\quad%
  \raisebox{0.2truecm}{%
   \begin{sideways}
    \tiny from: \url{www.scipy.org}
   \end{sideways}}
 \end{center}

 + SciKits and many other packages
\end{frame}

\begin{frame}
 \vspace*{-0.5truecm}
 \begin{columns}
  \begin{column}[t]{0.5\textwidth}
   \begin{center}
    \structure{www.scipy-lectures.org}

    \includegraphics[width=\textwidth]{ScipyLectCover}
   \end{center}
  \end{column}%
  \begin{column}[t]{0.5\textwidth}
   \begin{center}
    \structure{docs.scipy.org/doc/numpy/}

    \vspace{0.3truecm}
    \includegraphics[width=\textwidth]{NumpyManual}
   \end{center}
  \end{column}
 \end{columns}
\end{frame}

\begin{frame}{A wish list}
 \begin{itemize}
  \item we want to work with vectors and matrices
 \end{itemize}

 \begin{columns}
  \begin{column}{0.55\textwidth}
   \begin{displaymath}
    \begin{pmatrix}
     a_{11} & a_{12} & \cdots & a_{1n}\\
     a_{21} & a_{22} & \cdots & a_{2n}\\
     \vdots & \vdots & \ddots & \vdots\\
     a_{n1} & a_{n2} & \cdots & a_{nn}\\
    \end{pmatrix}
   \end{displaymath}
  \end{column}%
  \begin{column}{0.45\textwidth}
   \includegraphics[width=0.7\textwidth]{rgbarray}

   colour image as $N\times M\times3$-array
  \end{column}
 \end{columns}

 \begin{itemize}
  \item we want our code to run fast
  \item we want support for linear algebra
  \item \dots 
 \end{itemize}
\end{frame}

\begin{frame}{List indexing}
 \begin{center}
  \includegraphics[width=0.96\textwidth]{listindexing2}
 \end{center}
 \begin{itemize}
  \item indexing starts at 0
  \item negative indices count from the end of the list to the beginning
 \end{itemize}
\end{frame}

\begin{frame}{List slicing}

 basic syntax: \texttt{[start:stop:stride]}

 \vspace{0.2truecm}
 \begin{center}
  \includegraphics[width=\textwidth]{listindexing1}
 \end{center}
 \begin{itemize}
  \item slice contains the elements \texttt{start} to \texttt{stop-1}
  \item slice contains \texttt{stop-start} elements
  \item default values:
  \begin{itemize}
      \item \makebox[1.5truecm]{\texttt{start}\hfill} 0 (first element)
      \item \makebox[1.5truecm]{\texttt{stop}\hfill} N-1 (last element)
      \item \makebox[1.5truecm]{\texttt{stride}\hfill} 1
  \end{itemize}
 \end{itemize}
\end{frame}

\begin{frame}
 \begin{center}
  \structure{\large\bfseries Let's do some slicing}
 \end{center}
\end{frame}

\begin{frame}[fragile, t]{Matrices and lists of lists}
 
 \vspace{0.3truecm}
 Can we use lists of lists to work with matrices?

 \begin{columns}
  \begin{column}{0.4\textwidth}
   \begin{displaymath}
    \begin{pmatrix}
     0 & 1 & 2\\
     3 & 4 & 5\\
     6 & 7 & 8
    \end{pmatrix}
   \end{displaymath}
  \end{column}%
  \begin{column}{0.6\textwidth}
   \begin{center}
    \begin{lstlisting}[language=python]
matrix = [[0, 1, 2],
          [3, 4, 5],
          [6, 7, 8]]
    \end{lstlisting}

    \vspace{\baselineskip}
   \end{center}
  \end{column}
 \end{columns}
 \begin{itemize}
  \item How can we extract a row? \uncover<3>{\smiley}
  \item How can we extract a column? \uncover<3>{\frownie}
 \end{itemize}

 \vspace{0.3truecm}
 \begin{center}
  \only<2>{\structure{\large\bfseries Let's do some experiments}}%
  \only<3>{\alert{Lists of lists are not well suited to work with matrices}}
 \end{center}
\end{frame}

\end{document}
